\stepcounter{section}
\section*{\Large\centering {CHAPTER 3 \\ INTERNSHIP ACTIVITIES}}
\addcontentsline{toc}{section}{CHAPTER 3: Internship Activities}\label{3}

\subsection{Roles and Responsibilities}

During my internship at \textbf{Genese Solution Pvt.\ Ltd.}, I performed the following roles and responsibilities in the Cloud \& DevOps Engineering Department:

\begin{itemize}

    \item Assisted in designing and implementing \textbf{CI/CD pipelines} using AWS CodePipeline and AWS CodeBuild.
    
    \item Automated build, packaging, and deployment processes for serverless applications.
    
    \item Worked with \textbf{AWS SAM (Serverless Application Model)} for building and deploying Lambda-based applications.
    
    \item Developed and modified \textbf{AWS CloudFormation templates} to provision infrastructure using Infrastructure as Code (IaC).
    
    \item Managed deployment artifacts in \textbf{Amazon S3}, including packaging and version control of application builds.
    
    \item Configured and maintained \textbf{AWS IAM roles and policies} following the principle of least privilege.
    
    \item Assisted in creating and managing \textbf{serverless architectures using AWS Lambda} in a monorepo structure.
    
    \item Handled shared Lambda layers, versioning, and dependency management.
    
    \item Monitored application logs and system performance using \textbf{Amazon CloudWatch}.
    
    \item Identified and troubleshot build failures, deployment errors, and configuration issues.
    
    \item Supported multi-environment deployments (Development, Staging, Production).
    
    \item Participated in reviewing infrastructure changes and deployment workflows.
    
    \item Collaborated with development teams to improve automation and deployment efficiency.
    
    \item Maintained technical documentation related to CI/CD pipelines, IAM configuration, and deployment procedures.
    
    \item Followed Agile methodologies and participated in team meetings and progress discussions.
    
    \item Ensured adherence to cloud security best practices and operational standards.

\end{itemize}

\subsection{Weekly log}

\begin{table}[h!]
    \centering
    \caption{Weekly log of activities performed during the internship.}
\renewcommand{\arraystretch}{1.3}
\begin{tabular}{|c|p{13cm}|}
    
\hline
\textbf{Week} & \textbf{Activity Performed} \\
\hline

Week 1 & Project onboarding and AWS architecture review of GUMP Now (Lambda, RDS, EC2, CloudWatch, CI/CD). \\
\hline

Week 2 & Research on CloudWatch Synthetics canaries for availability, latency, and user-journey monitoring. \\
\hline

Week 3 & Design of CloudWatch on-call support dashboards for staging and production environments. \\
\hline

Week 4 & Implementation of CloudWatch dashboards and monitoring metrics for operational visibility. \\
\hline

Week 5 & Implementation of IAM-based authentication for RDS in a staging environment. \\
\hline

Week 6 & Research and execution of AWS SSM Patch Manager to patch all EC2 instances in staging. \\
\hline

Week 7 & Research on AWS WAF capabilities including managed rules and web attack mitigation strategies. \\
\hline

Week 8 & Design of automated CI/CD pipeline for Lambda monorepo using AWS SAM and custom trigger Lambda. \\
\hline

Week 9 & Implementation of automated Lambda monorepo pipeline with CodeBuild, SAM build, package, and deploy. \\
\hline

Week 10 & SMTP server proof of concept using Node.js \texttt{smtp-server} package for mail flow analysis. \\
\hline

Week 11 & Functional and usability testing of the official GUMP Now website. \\
\hline

Week 12 & Production improvements: Outlook invitation email fix and DKIM CSV download feature deployment to staging and production. \\
\hline

\end{tabular}
\end{table}

\subsection{Description of the Project Involved During Internship}

The \textbf{Genese Unified Mailing Platform (GUMP)} is an enterprise-grade cloud-based mailing solution developed by \textbf{Genese Solution Pvt.\ Ltd.} to streamline and centralize email communication processes for organizations. The platform is designed to handle high-volume email delivery requirements, including transactional emails, notification systems, marketing campaigns, and automated communication workflows. GUMP provides a unified interface for managing email templates, user segmentation, scheduling, tracking, and reporting, ensuring reliable and efficient communication between organizations and their customers. \cite{genese_gump_platform}

During my internship, I worked on multiple technical projects related to the \textbf{GUMP Now} platform, primarily focusing on cloud infrastructure, security enhancement, monitoring, automation, and operational support. One of the key contributions involved designing and implementing cloud monitoring and observability solutions using \textbf{AWS CloudWatch}, including dashboards and synthetic monitoring for improved availability tracking and proactive issue detection in staging and production environments.

I also contributed to cloud security improvements by implementing \textbf{IAM-based authentication for Amazon RDS} and strengthening system security through patch management of EC2 instances using \textbf{AWS SSM Patch Manager}. In addition, I designed and implemented an automated \textbf{CI/CD pipeline} for a Lambda monorepo using \textbf{AWS SAM} and a custom trigger Lambda function, enabling efficient and selective serverless deployments.

Furthermore, I supported email infrastructure and application-level enhancements, including an SMTP server proof of concept, resolving Outlook email rendering issues, implementing a DKIM CSV download feature, conducting website testing, and supporting both staging and production environments. These activities collectively aimed to enhance the reliability, security, and operational efficiency of the GUMP platform.


\subsection{Activities Performed}

\begin{itemize}

\item \textbf{Cloud Monitoring and Observability:}  
Researched and configured AWS CloudWatch features, including custom dashboards and performance metrics for staging and production environments. Evaluated CloudWatch Synthetics canaries to monitor application availability, response time, and end-to-end user journeys, supporting proactive incident detection and on-call operations.

\item \textbf{RDS IAM Authentication Implementation:}  
Implemented IAM-based authentication for Amazon RDS in the Gumpnow staging environment. Configured IAM roles and policies, updated database connection logic, and validated secure database access without static credentials.

\item \textbf{On-Call Support Dashboards:}  
Designed and deployed CloudWatch dashboards aggregating key service metrics such as health status, error rates, and alarms. Validated dashboards in both staging and production environments.

\item \textbf{Website and Application Testing:}  
Conducted functional and usability testing of the official GUMP Now website to ensure correct behavior, identify defects, and validate fixes before and after deployments.

\item \textbf{SMTP Server Proof of Concept:}  
Developed a proof-of-concept SMTP server using the Node.js \texttt{smtp-server} package to analyze email handling, message flow, and integration mechanisms within the email infrastructure.

\item \textbf{AWS WAF Research:}  
Researched AWS Web Application Firewall (WAF) capabilities, including managed rule groups and custom rules, to evaluate protection mechanisms against SQL injection, cross-site scripting (XSS), and other web vulnerabilities.

\item \textbf{Patch Management Using AWS SSM:}  
Patched all EC2 instances in the Gumpnow staging environment using AWS Systems Manager Patch Manager. Verified compliance, monitored execution, and ensured system stability after patching.

\item \textbf{Email Rendering and Deliverability Fixes:}  
Investigated and resolved an issue where invitation email buttons were not displayed correctly in Microsoft Outlook. Improved HTML/CSS compatibility and validated rendering across email clients.

\item \textbf{DKIM Feature Enhancement:}  
Implemented a feature in the \texttt{gumpnow-frontend-app} to enable downloading DKIM records in CSV format. Deployed and verified the feature in staging and production environments.

\item \textbf{CI/CD Automation for Lambda Monorepo:}  
Designed and implemented an automated CI/CD pipeline for a Lambda monorepo using AWS SAM. Developed a custom trigger Lambda function to detect code changes and selectively trigger deployments, improving efficiency and reducing unnecessary builds.

\end{itemize}

