
\section*{\Large\centering \textbf{CHAPTER 1 \\ INTRODUCTION }}
\addcontentsline{toc}{section}{CHAPTER 1: Introduction}
\setcounter{section}{1}

\subsection{Introduction}
Brain tumors are among the most lethal forms of cancer, contributing to high
mortality and morbidity rates worldwide. Early and accurate detection plays a
vital role in improving patient outcomes, as timely treatment significantly
enhances survival rates. However, traditional methods of diagnosis, such as
manual \gls{mri} analysis, prove to be time-consuming, laborious, and
error-prone. In recent years, significant progression has been witnessed in the
field of Artificial Intelligence, and deep learning, in particular, shows great
promise in overcoming these limitations.~\cite{razzak2018}

In the field of intelligent automation and high accuracy for discriminating
brain tumor presence from medical images, Convolutional Neural Networks, or
\gls{cnn}s, demonstrate remarkable potential. This research discusses the brain
tumor detection and classification problem. It provides an automated approach
to identify the position, size, and shape of tumors through deep learning
techniques applied to medical images. Numerous studies have been undertaken
using machine learning and deep learning techniques for this purpose, with
\gls{cnn}-based models consistently demonstrating some of the most effective
results in medical image analysis.~\cite{ilic2023}

\subsection{Problem Statement}
Brain tumors requiring timely diagnosis for effective treatment and improved survival rates. Conventional diagnostic methods, primarily reliant on manual analysis of \gls{mri} scans by radiologists, are time-consuming, prone to human error, and often lead to inconsistent results, especially in early detection stages. The rising volume of medical imaging data further exacerbates this issue, putting additional strain on healthcare systems. There is a compelling need for an automated, accurate, and efficient system that can assist in detecting and classifying brain tumors to support medical professionals in making informed decisions.\cite{aksoy2025}

% \subsection{Objectives}
%  To design and implement a convolutional neural network (\gls{cnn}) model, incorporating transfer learning with VGG16 to enhance accuracy, tailored for precise brain tumor detection and classification.

\subsection{Objectives}
The main objectives of this project are as follows:
\begin{itemize}
    \item To design and implement a convolutional neural network (\gls{cnn}) model.
    \item To incorporate transfer learning with VGG16 to enhance accuracy for precise brain tumor detection and classification.
\end{itemize}

\subsection{Scope and Limitation}
The scope of this research encompasses comprehensive preprocessing of medical
images, systematic training of the \gls{cnn} model on various available
datasets, and thorough evaluation of its performance using well-established
metrics including accuracy, precision, recall, and F1-score. This research
emphasizes the transformative role of artificial intelligence in reducing human
error and accelerating diagnostic processes, thereby significantly improving
patient outcomes and healthcare efficiency.

However, this research acknowledges several important limitations. The accuracy
and reliability of the system are inherently influenced by the quality,
diversity, and representativeness of the dataset used for training the neural
network. Given that the datasets employed in this research are limited to
available \gls{mri} scans, the variability may not fully capture the complete
spectrum of differences in tumor types, stages, or imaging conditions
encountered in diverse clinical scenarios.

Furthermore, this model is developed primarily for academic research purposes
and experimental validation. It should not be considered as a replacement for
professional medical judgment or clinical expertise. Additional considerations
include hardware computational constraints, model generalization capabilities
across different populations, and the complex ethical implications surrounding
the use of \gls{ai} in critical medical decision-making processes. These
considerations extend beyond the immediate scope of this study.

\subsection{Report Organization}
This research report is systematically organized into several comprehensive
chapters that present the research methodology, implementation details, and
findings in a logical progression. The structure is as follows:

\begin{itemize}

    \item \textbf{Chapter 1: Introduction} – Provides an overview of the research, outlining the background, objectives, significance, and scope of the study.

    \item \textbf{Chapter 2: Literature Review} – Provides an extensive review of existing approaches in brain tumor detection utilizing machine learning and deep learning techniques.

    \item \textbf{Chapter 3: System Analysis} – Presents detailed system analysis, including comprehensive requirement analysis and feasibility studies.

    \item \textbf{Chapter 4: System Design} – Covers the system design aspects, encompassing architectural framework decisions and model design considerations.

    \item \textbf{Chapter 5: Implementation and Testing} – Discusses implementation details and comprehensive testing procedures.

    \item \textbf{Chapter 6: Conclusion and Recommendations} – Concludes the research with key findings, practical recommendations, and directions for future research endeavors.

\end{itemize}