\newpage
\section{Introduction}
\subsection{Background}
The global incidence rate of brain and other \gls{cns} tumors is approximately 6.2 per 100,000 people per year, encompassing both malignant and benign tumors ~\cite{tisch2024}. In 2019, worldwide data recorded 347,992 new cases of brain cancer, with a gender distribution of 54\% males (187,491) and 46\% females (160,501) ~\cite{ilic2023}. The global \gls{asr} of brain cancer incidence was 4.8 per 100,000 in males, highlighting its significant prevalence across populations [2]. Brain and central nervous system cancers represent approximately 1.9\% of all cancer cases globally, ranking as the 19th most frequent malignancy and the 12th leading cause of cancer deaths (2.5\% of all cancers) ~\cite{ilic2023}.

Brain tumors are among the most lethal forms of cancer, with significant mortality and morbidity rates worldwide. Early and accurate detection is crucial for effective treatment and improved survival rates. Traditional diagnostic methods, such as manual \gls{mri} analysis, are time-consuming and prone to human error.
Recent advancements in deep learning, particularly \gls{cnn}s, have shown remarkable promise in automating and improving the accuracy of brain tumor detection from medical images ~\cite{razzak2018}.

Deep learning has emerged as a transformative technology in the field of brain tumor diagnosis and management. Brain tumor segmentation is the process of determining tumor location, size, and shape using automated methods, has become a prevalent application of artificial intelligence in medical imaging. Many researchers have explored various machine learning and deep learning approaches to optimize tumor detection, with \gls{cnn}s emerging as particularly promising methodologies.


